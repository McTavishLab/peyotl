\documentclass{bioinfo}
\copyrightyear{2014}
\pubyear{2014}
\usepackage{xspace}
\newcommand{\pey}{peyotl\xspace}
\newcommand{\ps}{phylesystem\xspace}
\newcommand{\otol}{Open Tree of Life\xspace}
\newcommand{\nexson}{otNexSON\xspace}
\newcommand{\mthcomment}[1]{{\color{red} \textsc{#1}}\xspace}
\newcommand{\ejmcomment}[1]{{\color{green} \textsc{#1}}\xspace}

\usepackage{natbib}
\bibliographystyle{apalike}

\usepackage{hyperref}
\hypersetup{colorlinks=true, linkcolor=black, citecolor=black, hyperindex=true, backref}
\begin{document}
\firstpage{1}
\title[peyotl python package]{peyotl: a Python library for interacting with Open Tree of Life data and web services}

\author[McTavish\textit{et~al}]{
    Emily Jane McTavish,$^{1,2}$
    Mark T.~Holder,$^{1,2}$\footnote{to whom correspondence should be addressed}~
    Ann Other Author,$^{3}$
    Ann Other Author,$^{4}$
    Ann Other Author,$^{5}$ $\ldots$
}
\address{$^{1}$Department of Ecology and Evolutionary Biology, University of Kansas, Lawrence KS, USA\\
$^{2}$Heidelberg Institute of Theoretical Studies, Heidelberg, Germany}

\history{Received on XXXXX; revised on XXXXX; accepted on XXXXX}

\editor{Associate Editor: XXXXXXX}

\maketitle

\begin{abstract}
\section{Summary:}
The Open Tree of Life project has written software infrastructure
to build and update a taxonomy and phylogenetic estimate that 
are intended to encompass all known species.
The project also has built infrastructure for curating and correcting
    archived versions of phylogenetic estimates.
That project has a service oriented architecture: the infrastructure consists of
    a large set of software tools that interact with each through web services.
Many of the data models used as arguments and responses for these services were designed
    by the project.
Here we describe \pey, a Python package that implements some parts of
    the \otol architecture and also provides simple routines for 
    using the web services and data types of the project.

The goals of \pey are to provide well-tested library that is used
    both on the server-side by components of \otol, but also
    available for client side use.
The library is intended to serve as an adaptor layer between \otol
    and other phylogenetic libraries written in Python (rather 
    than providing a full-feature phylogenetic library).

Source code available at \url{https://github.com/OpenTreeOfLife/peyotl}.
Documentation is available at \url{http://opentreeoflife.github.io/peyotl/}.
\section{Contact:} \href{mailto:mtholder@gmail.com}{mtholder@gmail.com}
\end{abstract}

\section{Introduction}
The \otol project can be a daunting resource for programmers to interact with.
Currently, the architecture provides separate tools for several task.
These include analytical software tools for:
\begin{enumerate}
    \item merging many taxonomies into a comprehensive taxonomy of life, and
    \item constructing a supertree for the millions of species on Earth from this 
            taxonomy and published phylogenetic estimates;
\end{enumerate}
data stores of:
\begin{enumerate}
   \setcounter{enumi}{2}
    \item the taxonomic names and synonymy information for the reference taxonomy,
    \item a collection of curated phylogenetic statements, and
    \item a graph database aligning the phylogenetic statements to each other and the taxonomy;
\end{enumerate}
web services for querying each of these data stores; and web applications to provide
user interfaces for
\begin{enumerate}
   \setcounter{enumi}{5}
    \item browsing and annotating the supertree estimate of the tree of life, and
    \item helping systematists ``curate'' published phylogenetic statements by providing
        metadata about studies, correcting the rooting of trees, and mapping the tips
        of the tree to a reference taxonomy.
\end{enumerate}
For the sake of efficiency, much of the communication between components of this architecture
use new data models for phylogenetics.

Here we describe a python library intended to make it easy for programmers outside
    of the \otol project to use the web services and data produced by the project without
    learning the details of all the data models and web-service API.
\pey is also used within the project, which should make it easier to keep the client library
     in sync with the services as new versions of the \otol APIs are created.


\begin{methods}
\section{Methods}
\end{methods}
\section{Results}

\section{Discussion and Conclusions}

\section{Conclusion}

\section*{Acknowledgement}

\paragraph{Funding\textcolon} We thank NSF AVATOL \#1208809, HITS, and an Alexander von Humboldt award to EJM for funding.
\bibliography{peyotl}
\end{document}
