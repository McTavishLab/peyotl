\documentclass{bioinfo}
\copyrightyear{2014}
\pubyear{2014}
\usepackage{xspace}
\newcommand{\pey}{peyotl\xspace}
\newcommand{\ps}{phylesystem\xspace}
\newcommand{\otol}{Open Tree of Life\xspace}
\newcommand{\nexson}{otNexSON\xspace}
\newcommand{\mthcomment}[1]{{\color{red} \textsc{#1}}\xspace}
\newcommand{\ejmcomment}[1]{{\color{green} \textsc{#1}}\xspace}

\usepackage{natbib}
\bibliographystyle{apalike}

\usepackage{hyperref}
\hypersetup{colorlinks=true, linkcolor=black, citecolor=black, hyperindex=true, backref}
\begin{document}
\firstpage{1}
\title[peyotl python package]{peyotl: a Python library for using Open Tree of Life data and web services}

\author[McTavish\textit{et~al}]{
    Emily Jane McTavish,$^{1,2}$
    Mark T.~Holder,$^{1,2}$\footnote{to whom correspondence should be addressed}~
    Ann Other Author,$^{3}$
    Ann Other Author,$^{4}$
    Ann Other Author,$^{5}$ $\ldots$
}
\address{$^{1}$Department of Ecology and Evolutionary Biology, University of Kansas, Lawrence KS, USA\\
$^{2}$Heidelberg Institute of Theoretical Studies, Heidelberg, Germany}

\history{Received on XXXXX; revised on XXXXX; accepted on XXXXX}

\editor{Associate Editor: XXXXXXX}

\maketitle

\begin{abstract}
\section{Summary:}
The Open Tree of Life project has written software infrastructure
to build and update a taxonomy and a phylogenetic estimate that 
are intended to encompass all known species.
The project also has built infrastructure for curating and correcting
    archived versions of phylogenetic estimates.
That project has a service oriented architecture: the infrastructure consists of
    a large set of software tools that interact with each through web services.
Many of the data models used as arguments and responses for these services were designed
    by the project.
Here we describe \pey, a Python package that implements some parts of
    the \otol architecture and also provides simple routines for 
    using the web services and data types of the project.

The goals of \pey are to provide well-tested library that is used
    both on the server-side by components of \otol, but also
    available for client side use.
The library is intended to serve as an adaptor layer between \otol
    and other phylogenetic libraries written in Python (rather 
    than providing a full-feature phylogenetic library).

%Source code available at \url{https://github.com/OpenTreeOfLife/peyotl}.
%Documentation is available at \url{http://opentreeoflife.github.io/peyotl/}.
\section{Contact:} \href{mailto:mtholder@gmail.com}{mtholder@gmail.com}
\end{abstract}

\section{Introduction}
The \otol project can be a daunting resource for programmers to interact with.
Currently, the architecture provides separate tools for several tasks.
These include analytical software tools for:
\begin{enumerate}
    \item merging many taxonomies into a comprehensive taxonomy of life, and
    \item constructing a supertree for the millions of species on Earth from this 
            taxonomy and published phylogenetic estimates;
\end{enumerate}
data stores of:
\begin{enumerate}
   \setcounter{enumi}{2}
    \item the taxonomic names and synonymy information for the reference taxonomy,
    \item a collection of curated phylogenetic statements, and
    \item a graph database aligning the phylogenetic statements to each other and the taxonomy;
\end{enumerate}
web services for querying each of these data stores; and web applications to provide
user interfaces for
\begin{enumerate}
   \setcounter{enumi}{5}
    \item browsing and annotating the supertree estimate of the tree of life, and
    \item helping systematists ``curate'' published phylogenetic statements by providing
        metadata about studies, correcting the rooting of trees, and mapping the tips
        of the tree to a reference taxonomy.
\end{enumerate}
For the sake of efficiency, much of the communication between components of this architecture
use new data models and formats.

Here we describe a python library intended to make it easy for programmers outside
    of the \otol project to use the web services and data produced by the project without
    learning the details of all the data models and web-service API.
\pey is also used within the project, which should make it easier to keep the client library
     in sync with the services as new versions of the \otol APIs are created.


\begin{methods}
\section{Methods}
\pey is written in pure Python.
It makes extensive use of subpackages to organize and isolate different components.
At the lowest level, the `phylo' and `utility' subpackages provide methods and classes
    for simple manipulations of phylogenetic and non-phylogenetic data which do not
    depend on the data models or services of the \otol project.
The NexSON formats\citep{NexSON}, a set of serializations of NeXML into JSON, are used
    within the project for storing curated phylogenetic estimates from the literature.
\pey supports: syntactic checking and conversions between these formats with a `nexson\_syntax'
    subpackage;
    validation of the content with a `nexson\_validation' subpackage; and 
    manipulation of instances of the NexSON data model with the `manip' subpackage.

The `phylesystem' subpackage contains the classes and functions used to implement the
    server side of the `phylesystem-api` services of the \otol project.
This component of the architecture \citep[described in more detail in][]{phylesystemapi}
    implements a create, delete, update operations for a collection of trees in NexSON
    format using the git versioning tool to maintain the history of edits to each
    tree.

The `api' subpackage of \pey provides classes that provide a more ``pythonic'' interface 
for \otol API calls (see Figure 1).
Each set of service calls is represented by a class which can perform necessary 
    format conversion of the arguments and responses so that the client code need 
    not be familiar with the details of each JSON response for the different API calls.
The use of \pey's API wrappers also allow client code to be more robust to changes in the API.
There have been 2 versions of the \otol API thus far.
\pey's wrappers allow the client code to specifically request a particular version of the API be used.
For API methods that have only changed in superficial ways (e.g. changes to the name of 
the arguments of the web services), the peyotl wrappers can format the client-supplied arguments
to either version of the web service calls.
Thus client code using \pey will not have to be updated if the current versions of the
    API are deprecated in favor of a richer version of the \otol API.
\pey can convert trees from \otol web services into the widely used Newick format with tip
    labels corresponding to either the Open Tree Taxonomy IDs or the taxonomic name of the taxon.
In the case of trees from published studies that are a part of the phylesystem corpus, the
    leaf labels can also be displayed using the original name strings that were found in
    the uploaded data.
This makes it feasible for client code to use the taxonomic assignments in phylesystem to 
    construct data sets that combine data from multiple studies.

Some nascent subpackages (`ott' and `gcmdr') of \pey are in development as re-implementations
    of other \otol tools (for taxonomic operations and driving of the ``treemachine'' supertree tool
    respectively).
These parts of \pey will provide a means of testing the canonical implementations of these
    tools, but will also offer more flexible interfaces for python programmers.

The software architecture of the \otol project was written to be modular and relatively 
    easy for any bioinformatician to deploy on their own computers.
Thus the endpoints of the \otol web services need not start with the \url{http://api.opentreeoflife.org} domain that the project supports.
To make \pey-based scripts configurable for different web-service endpoints at runtime, \pey
    makes use of INI-formatted configuration files.
This makes it easy to run service calls against the development domain (\url{http://devapi.opentreeoflife.org}), the aforementioned ``production'' services, or against locally installed
versions of the services without modifying the scripts involved (or cluttering each interaction with the web services with a set of arguments describing the endpoint to be used).

\end{methods}
\begin{figure}[htp]
\begin{verbatim}
from peyotl.sugar import oti, phylesystem_api
matches = oti.find_trees(ott_id=84761,
                         wrap_response=True)
newick = phylesystem_api.get(matches[0],
                             format='newick')
\end{verbatim}

\caption{An example of using the peyotl wrappers to search for trees containing a specific taxon, 
designated by the Open Tree Taxonomy ID, using the Open Tree Indexer (oti) tool.
The tree is then fetched in newick notation from the phylesystem web services.
Here the oti wrapper takes converts the python keyword argument `ott\_id' to
    the appropriate query for a web service call and wraps the list of dictionaries
    returned by the API call into a list of peyotl.api.TreeRef objects.
This object can be used as an argument to a phylesystem-api call and the wrapper around
    the phylesystem service converts the NexSON returned to a newick string.
}
\end{figure}

\section{Availability}
Snapshots are posted on the PyPI, which means that \pey can be installed using the pip
    installation tools via ``pip install peyotl'' from a terminal.
We use GitHub (\url{https://github.com/OpenTreeOfLife/peyotl}) for version control, and
    that site has all of the working branches of \pey.
The documentation, included some a brief tutorial, is hosted at \url{http://opentreeoflife.github.io/peyotl}


\section*{Acknowledgement}

\paragraph{Funding\textcolon} We thank NSF AVATOL \#1208809, HITS, and an Alexander von Humboldt award to EJM for funding.
\bibliography{peyotl}
\end{document}
